\documentclass{article}
\usepackage[margin=2cm]{geometry}
\usepackage{amsmath}
\usepackage{amssymb}
\newcommand\abs[1]{\left|#1\right|}
\usepackage{nicefrac}
\usepackage{amsthm}
\author{Uri Goodman - 315554907}
\title{Introduction to Error Correcting Codes - H.W. 1}
\begin{document}
    \maketitle
    \begin{enumerate}
        \item We need to prove is that the basis for $C_k$  is perpendicular to the basis for $C_{q-k+1}$. A basis for \[C_k = Sp\left\{ \begin{pmatrix} a_1 ^0\\\vdots\\a_q ^0\end{pmatrix},\begin{pmatrix}
                a_1 ^1\\\vdots\\a_q ^1
            \end{pmatrix},\dots,\begin{pmatrix}
                a_1 ^{k-1}\\\vdots\\a_q ^{k-1}
            \end{pmatrix}\right\}\qquad C_{q-k+1} = Sp\left\{ \begin{pmatrix} a_1 ^0\\\vdots\\a_q ^0\end{pmatrix},\begin{pmatrix}
                a_1 ^1\\\vdots\\a_q ^1
            \end{pmatrix},\dots,\begin{pmatrix}
                a_1 ^{q-k+1}\\\vdots\\a_q ^{q-k+1}
        \end{pmatrix}\right\}\] Let $ a \in C_k$ and  $b\cdot C_{q-k+1}$. Thus $a\bullet b = a_1 ^r +a_2 ^r + \dots +a_q ^r$. Where $r < q-1$
        \begin{itemize}
            \item If $r=0$ then $a\bullet b =1\cdot q= p^l =0$ since the characteristic of a field of order $q=p^l$ is $p$. \\
            \item  If $r>0$ then $q >2$. We will use the second hint, that there exists a $\alpha \in \mathbb F _q$ such that $1,\alpha,\alpha^2,\dots,\alpha^{q-2}\neq 0$. Let $a\bullet b = \beta, \beta \in \mathbb F_q$ then multiplying both sides by $\alpha^r$ we have $\left(a_1\cdot \alpha\right) ^r +\left(a_2\cdot \alpha\right) ^r + \dots +\left(a_q\cdot\alpha\right) ^r = \alpha^r \cdot \beta$. Also, multiplying by $\alpha$ gives us an invertible function meaning that the set $\left\{
                a_1,a_2,\dots,a_n \right\}$ is equivalent to the set $\left\{ a_1\cdot \alpha,a_2\cdot \alpha,\dots,a_q\cdot \alpha \right\}$. Giving us the result that \[a_1 +^r a_2 ^r+\dots +a_q ^r= \frac{(\alpha a_1)^r + (\alpha a_2)^r +\dots+ (\alpha a_q)^r }{\alpha ^r}\Rightarrow  \beta = \beta \cdot \alpha^r\Rightarrow \beta = 0\] Since $\alpha ^r \neq 0$.
        \end{itemize}
    Meaning that in all cases, the dot product of a vector in $C_k$ and a vector in $C_{q-k+1}$.
        \item 
        \item We showed in class that the distance is equivalent to the minimal distance from zero. We will mark this word as $w$. Puncturing the code in all of the places that $w$ doesn't vanish is a linear transformation since we are just deleting columns from the generator matrix. The kernel has to have a dimension of 1. It doesn't equal to 0 since $w$ is in it and it isn't larger than one since
    \end{enumerate}<++>
\end{document}
